\documentclass[12pt,]{article}
\usepackage[utf8]{inputenc}
\usepackage[T1]{fontenc}
\usepackage{mathptmx}
\usepackage{geometry}
\usepackage{mathtools}
\usepackage[english]{babel}
\usepackage{graphicx}
\usepackage[os=win]{menukeys}
\usepackage[figurename=Gambar]{caption}
\usepackage{hyperref}
\usepackage{minted}
\usepackage[yyyymmdd,hhmmss]{datetime}

\newcommand{\WindowsLogo}{\raisebox{-0.1em}{\includegraphics[height=0.8em]{winlogo/Windows_3_logo_simplified}}}
\newcommand{\PowerLogo}{\raisebox{-0.1em}{\includegraphics[height=0.8em]{winlogo/power}}}
\newcommand{\WinKey}{\keys{\WindowsLogo}}	
\newcommand{\PowerKey}{\keys{\PowerLogo}}

\newcommand{\ShowOsVersion}{
	\immediate\write18{\unexpanded{foo=`uname -sro` && echo "\\\verb${foo}" > tmp.tex}}
	\input{tmp}\immediate\write18{rm tmp.tex}
}

\newcommand{\ShowTexVersion}{
	\immediate\write18{\unexpanded{foo=`pdflatex -version | head -n1` && echo "\\\verb${foo}" > tmp.tex}}
	\input{tmp}\immediate\write18{rm tmp.tex}
}	

\addto\captionsenglish{\renewcommand{\contentsname}{Daftar Isi}}

\hypersetup{
	colorlinks=true, %set true if you want colored links
	linktoc=all,     %set to all if you want both sections and subsections linked
	linkcolor=blue,  %choose some color if you want links to stand out
}

\geometry{
	a4paper,
	left=15mm,
	right=10mm,
	top=10mm,
	bottom=10mm,
}

\title{\Large \bf
	Tutorial Penggunaan HTS dan Grapheme
}

\author{Achmadi ST MT}

\date{}

\begin{document}
	\maketitle
	\thispagestyle{empty}
	\pagestyle{empty}
	
	\vspace*{550px}
	\noindent This book written using: \\
	OS : \ShowOsVersion \\
	TeX : \ShowTexVersion \\
	Update: {\today} at \currenttime \\
	
	\noindent Document Tex Source:\\
	\url{https://github.com/mekatronik-achmadi/hts_grapheme/blob/master/panduan/panduan.tex}
	
	\newpage
	(catatan: Daftar Isi dan Index bisa diklik)
	\tableofcontents
	
	\newpage
	\section{Requirement}
	
	Untuk menjalankan isi panduan ini diperlukan antara lain:
	
	\subsection{Sistem Komputer}
	Di sisi sistem komputer, membutuhkan minimal:
	
	\begin{itemize}
		\item Komputer PC atau Laptop dengan pendingin bagus.
		\item Multi-Core Processor dengan clock minimal 2.5GHz
		\item RAM PC3 dengan kapasitas minimal 8GB
		\item Sistem Operasi berbasis Linux atau Unix-\textit{like}.\\
		Direkomendasikan Arch-Linux atau Ubuntu-MATE.
		Tidak direkomendasikan MacOS atau BSD-\textit{family} lainnya.
		\item Koneksi internet minimal stabil 1MBps (untuk download sources).
	\end{itemize}
	
	\subsection{Pengguna}
	Sedangkan untuk pengguna sendiri, membutuhkan minimal:
	
	\begin{itemize}
		\item Memahami akses shell atau terminal ke suatu alamat directory/folder.
		\item Memahami struktur directory/folder dan paham pindah alamatnya.
		\item Cukup familiar dengan bahasa Unix shell (Bash).
		\item Cukup familiar dengan instalasi paket di distro yang digunakan.
		\item Cukup kenal dengan bahasa bantu Perl dan Python.
		\item Tidak alergi pemrograman.
		\item Paham menggunakan Google dalam English.
		\item Tidak malas untuk troubleshot sendiri.
	\end{itemize}

	Jika semua kebutuhan ini terpenuhi, maka kemungkinan besar anda mampu menggunakan panduan ini.
	Jika tidak, jangan menyerah, semua kesulitan akan berakhir, baik dengan memang selesai dan sukses, atau tinggalkan saja jika sudah menyerah.
	
	\newpage
	\section{Instalasi Tools}.
	
	Berikut adalah instalasi semua tools yang dibutuhkan secara runtut.
	
	\subsection{Download Sources}
	
	Berikut download URL (update Juni 2019) untuk semua sources yang dibutuhkan.
	Semua teks link dibawah ini dapat diklik dan otomatis membuka webrowser menuju URL yang diklik.
	
	
	\subsubsection{Free Open-Source}
	
	Semua paket open-sources berikut dapat didownload dengan bebas.
	
	\begin{itemize}
		\item festival-2.5.0-release.tar.gz.\\
		\url{http://www.festvox.org/packed/festival/2.5/festival-2.5.0-release.tar.gz}
		
		\item speech\_tools-2.5.0-release.tar.gz \\
		\url{http://www.festvox.org/packed/festival/2.5/speech_tools-2.5.0-release.tar.gz}
		
		\item festvox-2.8.0-release.tar.gz.\\
		\url{http://www.festvox.org/packed/festvox/2.8/festvox-2.8.0-release.tar.gz}
		
		\item festlex\_CMU.tar.gz, festlex\_OALD.tar.gz, dan festlex\_POSLEX.tar.gz. \\
		\url{http://www.festvox.org/packed/festival/2.5/festlex_CMU.tar.gz}\\
		\url{http://www.festvox.org/packed/festival/2.5/festlex_OALD.tar.gz}\\
		\url{http://www.festvox.org/packed/festival/2.5/festlex_POSLEX.tar.gz}
		
		\item festvox\_kallpc16k.tar.gz dan festvox\_rablpc16k.tar.gz \\
		\url{http://www.festvox.org/packed/festival/2.5/voices/festvox_kallpc16k.tar.gz}\\
		\url{http://www.festvox.org/packed/festival/2.5/voices/festvox_rablpc16k.tar.gz}
		
		\item festvox\_don.tar.gz dan  festvox\_kedlpc16k.tar.gz\\
		\url{http://www.cstr.ed.ac.uk/downloads/festival/1.95/festvox_don.tar.gz} \\
		\url{http://www.cstr.ed.ac.uk/downloads/festival/1.95/festvox_kedlpc16k.tar.gz}
		
		\item festvox\_cmu\_us\_*\_cg.tar.gz. \\
		\url{http://www.festvox.org/packed/festival/2.5/voices/festvox_cmu_us_aew_cg.tar.gz}\\
		\url{http://www.festvox.org/packed/festival/2.5/voices/festvox_cmu_us_ahw_cg.tar.gz}\\
		\url{http://www.festvox.org/packed/festival/2.5/voices/festvox_cmu_us_aup_cg.tar.gz}\\
		\url{http://www.festvox.org/packed/festival/2.5/voices/festvox_cmu_us_awb_cg.tar.gz}\\
		\url{http://www.festvox.org/packed/festival/2.5/voices/festvox_cmu_us_axb_cg.tar.gz}\\
		\url{http://www.festvox.org/packed/festival/2.5/voices/festvox_cmu_us_bdl_cg.tar.gz}\\
		\url{http://www.festvox.org/packed/festival/2.5/voices/festvox_cmu_us_clb_cg.tar.gz}\\
		\url{http://www.festvox.org/packed/festival/2.5/voices/festvox_cmu_us_eey_cg.tar.gz}\\
		\url{http://www.festvox.org/packed/festival/2.5/voices/festvox_cmu_us_fem_cg.tar.gz}\\
		\url{http://www.festvox.org/packed/festival/2.5/voices/festvox_cmu_us_gka_cg.tar.gz}\\
		\url{http://www.festvox.org/packed/festival/2.5/voices/festvox_cmu_us_jmk_cg.tar.gz}\\
		\url{http://www.festvox.org/packed/festival/2.5/voices/festvox_cmu_us_ksp_cg.tar.gz}\\
		\url{http://www.festvox.org/packed/festival/2.5/voices/festvox_cmu_us_ljm_cg.tar.gz}\\
		\url{http://www.festvox.org/packed/festival/2.5/voices/festvox_cmu_us_lnh_cg.tar.gz}\\
		\url{http://www.festvox.org/packed/festival/2.5/voices/festvox_cmu_us_rms_cg.tar.gz}\\
		\url{http://www.festvox.org/packed/festival/2.5/voices/festvox_cmu_us_rxr_cg.tar.gz}\\
		\url{http://www.festvox.org/packed/festival/2.5/voices/festvox_cmu_us_slp_cg.tar.gz}\\
		\url{http://www.festvox.org/packed/festival/2.5/voices/festvox_cmu_us_slt_cg.tar.gz}
		
		\item HTS-2.3\_for\_HTK-3.4.1.tar.bz2 \\
		\url{http://hts.sp.nitech.ac.jp/archives/2.3/HTS-2.3_for_HTK-3.4.1.tar.bz2}
		
		\item hts\_engine\_API-1.10.tar.gz \\
		\url{http://downloads.sourceforge.net/hts-engine/hts_engine_API-1.10.tar.gz}
		
		\item SPTK-3.10.tar.gz \\
		\url{http://downloads.sourceforge.net/sp-tk/SPTK-3.10.tar.gz}

		\item flite-2.0.0-release.tar.bz2 \\
		\url{http://festvox.org/flite/packed/flite-2.0/flite-2.0.0-release.tar.bz2}
	\end{itemize}
	
	\subsubsection{Limited Open-Sources}
	
	Semua paket open-sources berikut dapat didownload hanya dengan hak akses khusus.
	
	\begin{itemize}
		\item HTK-3.4.1.tar.gz \\
		\url{http://htk.eng.cam.ac.uk/ftp/software/HTK-3.4.1.tar.gz}
		
		\item HTK-samples-3.4.1.tar.gz \\
		\url{http://htk.eng.cam.ac.uk/ftp/software/HTK-samples-3.4.1.tar.gz}
		
		\item HDecode-3.4.1.tar.gz \\
		\url{http://htk.eng.cam.ac.uk/prot-docs/hdecode.shtml}
		
		
	\end{itemize}

	\subsection{Instalasi Paket Dasar}
	
	Selanjutnya adalah instalasi kebutuhan dasar yang telah tersedia di repository distro yang anda pakai.
	Paket yang dibutuhkan adalah:
	
	\begin{itemize}
		\item compiler-toolchain atau build-system.
		\item editor cli gawk.
		\item alternatif shell csh.
		\item pustaka pemrograman libx11.
		\item pustaka pemrograman ncurses5.
		\item sound processor sox.
	\end{itemize}

	Untuk instalasi ini, anda mungkin akan butuh koneksi internet dan hak akses root/sudo. \\
	
	Berikut perintah untuk instalasi di distro Ubuntu:
	\begin{minted}[frame=lines,fontsize=\footnotesize]{bash}
sudo apt-get install csh gawk sox libx11-dev gcc-multilib libncurses5-dev
	\end{minted}
	
	Untuk distro Arch-Linux:
	\begin{minted}[frame=lines,fontsize=\footnotesize]{bash}
sudo pacman -S tcsh gawk sox libx11 gcc-multilib 
	\end{minted}
	
	ditambah paket ncurses5-compat-libs dari AUR:\\
	\url{https://aur.archlinux.org/packages/ncurses5-compat-libs/}
	
	\begin{minted}[frame=lines,fontsize=\footnotesize]{bash}
sudo pacman -U ncurses5-compat-libs*
	\end{minted}
	
	\newpage
	\subsection{Persiapan}
	
	Berikut langkah persiapan instalasi.
	
	\begin{enumerate}
		\item buat folder yang berisi semua paket source yang telah dikumpulkan.
		Contoh disini adalah di alamat \textbf{\textasciitilde/HTS/install}.
		
		\item Salin semua paket source ke dalam folder tersebut.
		
		\item Tentukan alamat folder tujuan instalasi.
		Contoh disini adalah di alamat \textbf{\textasciitilde/.hts\_sptk}
		
		\item Buka jendela terminal baru.
		
		\item Masukkan variabel lingkungan untuk alamat paket sources dan alamat tujuan instalasi.
		\begin{minted}[frame=lines,fontsize=\footnotesize]{bash}
export SOURCE_DIR=~/HTS/install
export TARGET_DIR=~/.hts_sptk
		\end{minted}
		
		\item Check file di folder paket sumber.
		\begin{minted}[frame=lines,fontsize=\footnotesize]{bash}
ls -1 $SOURCE_DIR
		\end{minted}
		
		dimana hasil yang diharapkan.
	\begin{minted}[frame=lines,fontsize=\footnotesize]{bash}
festival-2.5.0-release.tar.gz
festlex_CMU.tar.gz
festlex_OALD.tar.gz
festlex_POSLEX.tar.gz
festvox-2.8.0-release.tar.gz
festvox_cmu_us_ahw_cg.tar.gz
festvox_cmu_us_aup_cg.tar.gz
festvox_cmu_us_awb_arctic_hts.tar.gz
festvox_cmu_us_awb_cg.tar.gz
festvox_cmu_us_axb_cg.tar.gz
festvox_cmu_us_bdl_cg.tar.gz
festvox_cmu_us_clb_cg.tar.gz
festvox_cmu_us_fem_cg.tar.gz
festvox_cmu_us_gka_cg.tar.gz
festvox_cmu_us_jmk_cg.tar.gz
festvox_cmu_us_ksp_cg.tar.gz
festvox_cmu_us_rms_cg.tar.gz
festvox_cmu_us_rxr_cg.tar.gz
festvox_cmu_us_slt_cg.tar.gz
festvox_don.tar.gz
festvox_kallpc16k.tar.gz
festvox_kedlpc16k.tar.gz
festvox_rablpc16k.tar.gz
flite-2.0.0-release.tar.bz2
HDecode-3.4.1.tar.gz
HTK-3.4.1.tar.gz
HTK-samples-3.4.1.tar.gz
HTS-2.3_for_HTK-3.4.1.tar.bz2
hts_engine_API-1.10.tar.gz
speech_tools-2.5.0-release.tar.gz
SPTK-3.10.tar.gz
	\end{minted}
	
	\item buat folder tujuan.
	\begin{minted}[frame=lines,fontsize=\footnotesize]{bash}
mkdir -p $TARGET_DIR
	\end{minted}
	
	\item Anda dapat menutup jendela terminal/shell.
		
	\end{enumerate}

	\newpage
	\subsection{Extraksi}
	
	Selanjutnya adalah ekstraksi semua paket sources.
	Berikut langkahnya:
	
	\begin{enumerate}
		\item Buka jendela terminal baru.
		
		\item Masukkan variabel lingkungan untuk alamat paket sources dan alamat tujuan instalasi.
		\begin{minted}[frame=lines,fontsize=\footnotesize]{bash}
export SOURCE_DIR=~/HTS/install
export TARGET_DIR=~/.hts_sptk
		\end{minted}
		
		\item Masukkan perintah ekstraksi untuk semua paket:
		\begin{minted}[frame=lines,fontsize=\footnotesize]{bash}
tar zxvf $SOURCE_DIR/SPTK-3.10.tar.gz -C $TARGET_DIR
tar zxvf $SOURCE_DIR/HTK-3.4.1.tar.gz -C $TARGET_DIR
tar zxvf $SOURCE_DIR/HDecode-3.4.1.tar.gz -C $TARGET_DIR
tar zxvf $SOURCE_DIR/HTK-samples-3.4.1.tar.gz -C $TARGET_DIR
tar jxvf $SOURCE_DIR/HTS-2.3_for_HTK-3.4.1.tar.bz2 -C $TARGET_DIR
tar zxvf $SOURCE_DIR/hts_engine_API-1.10.tar.gz -C $TARGET_DIR
tar zxvf $SOURCE_DIR/speech_tools-2.5.0-release.tar.gz -C $TARGET_DIR
tar zxvf $SOURCE_DIR/festival-2.5.0-release.tar.gz -C $TARGET_DIR
tar zxvf $SOURCE_DIR/festlex_CMU.tar.gz -C $TARGET_DIR
tar zxvf $SOURCE_DIR/festlex_POSLEX.tar.gz -C $TARGET_DIR
tar zxvf $SOURCE_DIR/festlex_OALD.tar.gz -C $TARGET_DIR
tar zxvf $SOURCE_DIR/festvox_kallpc16k.tar.gz -C $TARGET_DIR
tar zxvf $SOURCE_DIR/festvox_rablpc16k.tar.gz -C $TARGET_DIR
tar zxvf $SOURCE_DIR/festvox_kedlpc16k.tar.gz -C $TARGET_DIR
tar zxvf $SOURCE_DIR/festvox_cmu_us_ahw_cg.tar.gz -C $TARGET_DIR
tar zxvf $SOURCE_DIR/festvox_cmu_us_aup_cg.tar.gz -C $TARGET_DIR
tar zxvf $SOURCE_DIR/festvox_cmu_us_awb_arctic_hts.tar.gz -C $TARGET_DIR
tar zxvf $SOURCE_DIR/festvox_cmu_us_awb_cg.tar.gz -C $TARGET_DIR
tar zxvf $SOURCE_DIR/festvox_cmu_us_axb_cg.tar.gz -C $TARGET_DIR
tar zxvf $SOURCE_DIR/festvox_cmu_us_bdl_cg.tar.gz -C $TARGET_DIR
tar zxvf $SOURCE_DIR/festvox_cmu_us_clb_cg.tar.gz -C $TARGET_DIR
tar zxvf $SOURCE_DIR/festvox_cmu_us_fem_cg.tar.gz -C $TARGET_DIR
tar zxvf $SOURCE_DIR/festvox_cmu_us_gka_cg.tar.gz -C $TARGET_DIR
tar zxvf $SOURCE_DIR/festvox_cmu_us_jmk_cg.tar.gz -C $TARGET_DIR
tar zxvf $SOURCE_DIR/festvox_cmu_us_ksp_cg.tar.gz -C $TARGET_DIR
tar zxvf $SOURCE_DIR/festvox_cmu_us_rms_cg.tar.gz -C $TARGET_DIR
tar zxvf $SOURCE_DIR/festvox_cmu_us_rxr_cg.tar.gz -C $TARGET_DIR
tar zxvf $SOURCE_DIR/festvox_cmu_us_slt_cg.tar.gz -C $TARGET_DIR
tar zxvf $SOURCE_DIR/festvox-2.8.0-release.tar.gz -C $TARGET_DIR
tar jxvf $SOURCE_DIR/flite-2.0.0-release.tar.bz2 -C $TARGET_DIR
ls -1 $TARGET_DIR
		\end{minted}
		
		\item setelah baris perintah terakhir dieksekusi, maka hasil yang diharapkan:
		\begin{minted}[frame=lines,fontsize=\footnotesize]{bash}
ChangeLog
COPYING
festival
festvox
flite-2.0.0-release
htk
HTS-2.3_for_HTK-3.4.1.patch
HTS_Document.pdf
hts_engine_API-1.10
INSTALL
README
samples
speech_tools
SPTK-3.10
		\end{minted}
		
		\item anda dapat menutup jendela terminal/shell.
		
	\end{enumerate}
	
	\newpage
	\subsection{Build dan Install}
	
	Selanjutnya adalah langkah instalasi setiap komponen.
	
	\begin{enumerate}
		\item Buka jendela terminal baru.
		
		\item Masukkan variabel lingkungan untuk alamat paket sources dan alamat tujuan instalasi.
		\begin{minted}[frame=lines,fontsize=\footnotesize]{bash}
export SOURCE_DIR=~/HTS/install
export TARGET_DIR=~/.hts_sptk
		\end{minted}
		
		\item Jendela teminal/shell ini yang akan digunakan untuk instalasi semua komponen.
		Jangan menutup atau membuka jendela baru.
		
		\item Berikutnya perintah instalasi tiap komponen.
	\end{enumerate}

	\subsubsection{SPTK}
	\begin{minted}[frame=lines,fontsize=\footnotesize]{bash}
cd $TARGET_DIR/SPTK-3.10
./configure --prefix=$TARGET_DIR
make all
make install
	\end{minted}
	
	\subsubsection{HTK}
	
	Untuk Ubuntu:
	\begin{minted}[frame=lines,fontsize=\footnotesize]{bash}
cd $TARGET_DIR/htk
patch -p1 < ../HTS-2.3_for_HTK-3.4.1.patch
./configure CFLAGS=-DARCH=linux --prefix=$TARGET_DIR
make all
make install
make hdecode
make install-hdecode
	\end{minted}
	
	Untuk Arch-Linux:
	\begin{minted}[frame=lines,fontsize=\footnotesize]{bash}
cd $TARGET_DIR/htk
patch -p1 < ../HTS-2.3_for_HTK-3.4.1.patch
./configure --prefix=$TARGET_DIR
make all
make install
make hdecode
make install-hdecode
	\end{minted}
	
	\subsubsection{HTS}
	\begin{minted}[frame=lines,fontsize=\footnotesize]{bash}
cd $TARGET_DIR/hts_engine_API-1.10
./configure --prefix=$TARGET_DIR
make all
make install
	\end{minted}
	
	\subsubsection{Flite}
	\begin{minted}[frame=lines,fontsize=\footnotesize]{bash}
cd $TARGET_DIR/flite-2.0.0-release
./configure --prefix=$TARGET_DIR
make all
make install
	\end{minted}
	
	\newpage
	\subsubsection{Speech-Tools}
	\begin{minted}[frame=lines,fontsize=\footnotesize]{bash}
cd $TARGET_DIR/speech_tools
./configure --prefix=$TARGET_DIR
make all
make install
	\end{minted}
	
	\subsubsection{Festival}
	\begin{minted}[frame=lines,fontsize=\footnotesize]{bash}
cd $TARGET_DIR/festival
./configure --prefix=$TARGET_DIR
make all
make install
	\end{minted}
	
	\subsubsection{FestVox}
	\begin{minted}[frame=lines,fontsize=\footnotesize]{bash}
cd $TARGET_DIR/festvox
./configure --prefix=$TARGET_DIR
sed -i "s#NUL)#NULL)#g" src/vc/src/sub/gmm_sub.cc
make all
	\end{minted}
	
	Instalasi seluruh komponen telah selesai, maka anda menutup jendela terminal/shell.
	
	\subsection{Tools EnVars}
	
	Selanjutnya untuk dapat menggunakan tools yang telah diinstal,
	anda perlu men-set variable lingkungan (Environment Variables atau EnVars),
	sehingga shell/terminal dapat mengakses tools tersebut.
	
	Jika contoh dalam panduan ini menggunakan alamat instalasi \textbf{\textasciitilde/.hts\_sptk},
	maka perintah set envars:
	
	\begin{minted}[frame=lines,fontsize=\footnotesize]{bash}
export TARGET_DIR=~/.hts_sptk

export PATH=$TARGET_DIR/bin:$PATH
export PATH=$TARGET_DIR/festival/bin:$PATH
export PATH=$TARGET_DIR/speech_tools/bin:$PATH

export FESTVOXDIR=$TARGET_DIR/festvox
export FESTDIR=$TARGET_DIR/festival
export ESTDIR=$TARGET_DIR/speech_tools

export PATH=$FESTDIR/examples:$PATH
	\end{minted}
	
	Perintah ini wajib dijalankan setiap kali anda menggunakan terminal/shell yang baru untuk menjalankan tools yang telah diinstal.
	
\end{document}
